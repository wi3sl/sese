\documentclass[11pt]{scrartcl}

\usepackage{ucs}
\usepackage[utf8x]{inputenc}
\usepackage{ngerman}
\usepackage{amsmath,amssymb,amstext}
\usepackage{graphicx}
\usepackage[automark]{scrpage2}

\pagestyle{scrheadings}

\title{Product Backlog \\ \large Name des Projektes}
\author{Alexander Duml, Stefan Müller, Thomas Perl \& Martin Wieser}
\date{\today in Wien}

\begin{document}

\maketitle

\section{Userstories}

Template: Als $<ROLLE>$ will ich $<ZIEL>$, so dass $<BEGR"UNDUNG (optional)>$.

\subsection{Reservierung vornehmen $\surd$}

Als Rezeptionist will ich schnell und einfach neue Reservierungen anlegen, so dass ich mich wieder den Kunden widmen kann.

\subsection{Kundendaten einsehen $\surd$}

Als Geschäftsführer will ich Einblick in die Kundendaten, so dass ich mit ihnen Kontakt aufnehmen kann.

\subsection{Rechnung erstellen $\surd$}

Als Rezeptionist will ich Rechnungen erstellen, so dass ich diese den Kunden vorlegen kann.

\subsection{Reservierung stornieren $\surd$}

Als Rezeptionist will ich Reservierungen stornieren.

\subsection{Frühzeitige Abreise erfassen $\surd$}

Als Rezeptionist will ich ein frühzeitige Abreise erfassen, so dass die Zimmer wieder als "frei" erkannt werden.

\subsection{Zimmerbelegung anzeigen}

Als Geschäftsführer will ich in die Zimmerbelegung einsehen.

\subsection{Rechnungen anzeigen $\surd$}

Als Geschäftsführer möchte ich mir schnell und einfach alle ausgestellten und noch offenen Rechnungen anzeigen lassen.

\subsection{Putzplan erstellen}

Als Putzfrau möchte ich einen Putzplan erstellen, der alle zu putzenden Zimmer auflistet.

\subsection{Kunde anlegen $\surd$}

Als Rezeptionist will ich neue Kunden erstellen können, so dass ich ihre Daten für die Rechnung habe.

\subsection{Zimmer anlegen $\surd$}

Als Geschäftsfüherer will ich neue Zimmer hinzufügen, so dass das System nach einen Hotelausbau korrekt läuft.

\subsection{Daten löschen $\surd$}

Als Geschäftsfüherer will ich alte Daten löschen können, so dass das System nach einen Hotelumbau korrekt läuft.

\subsection{Daten bearbeiten $\surd$}

Als Geschäftsfüherer will ich Daten ändern können, so dass fehlerhafte Eingabe korrigiert oder Informationen ergänzt werden können.

\subsection{Reservierung mit mehreren Zimmern $\surd$}

Als Rezeptionist will ich Reservierungen mit mehreren Zimmern anlegen, so dass bei einer Stornierung alle Zimmer frei werden.

\subsection{Reservierung mit mehreren Kunden $\surd$}

Als Rezeptionist will ich Reservierungen mit mehreren Kunden anlegen, so dass zukünftige Discounts korrekt berechnet werden.

\section{Sprint}

\subsection{Sprint 1}

\subsubsection{Kunde anlegen}

Als Rezeptionist will ich neue Kunden erstellen können, so dass ich ihre Daten für die Rechnung habe.

\subsubsection{Zimmer anlegen}

Als Geschäftsfüherer will ich neue Zimmer hinzufügen, so dass das System nach einen Hotelausbau korrekt läuft.

\subsubsection{Reservierung vornehmen}

Als Rezeptionist will ich schnell und einfach neue Reservierungen anlegen, so dass ich mich wieder den Kunden widmen kann.

\subsubsection{Daten löschen}

Als Geschäftsfüherer will ich alte Daten löschen können, so dass das System nach einen Hotelumbau korrekt läuft.

\subsubsection{Daten bearbeiten}

Als Geschäftsfüherer will ich Daten ändern können, so dass fehlerhafte Eingabe korrigiert oder Informationen ergänzt werden können.

\subsection{Sprint 2}

\subsection{Reservierung mit mehreren Zimmern}

Als Rezeptionist will ich Reservierungen mit mehreren Zimmern anlegen, so dass bei einer Stornierung alle Zimmer frei werden.

\subsection{Reservierung mit mehreren Kunden}

Als Rezeptionist will ich Reservierungen mit mehreren Kunden anlegen, so dass zukünftige Discounts korrekt berechnet werden.

\subsection{Sprint 3}

\subsection{Kundendaten einsehen}

Als Geschäftsführer will ich Einblick in die Kundendaten, so dass ich mit ihnen Kontakt aufnehmen kann.

\subsection{Rechnung erstellen}

Als Rezeptionist will ich Rechnungen erstellen, so dass ich diese den Kunden vorlegen kann.

\subsection{Reservierung stornieren}

Als Rezeptionist will ich Reservierungen stornieren.

\subsection{Frühzeitige Abreise erfassen}

Als Rezeptionist will ich ein frühzeitige Abreise erfassen, so dass die Zimmer wieder als "frei" erkannt werden.


\subsection{Rechnungen anzeigen}

Als Geschäftsführer möchte ich mir schnell und einfach alle ausgestellten und noch offenen Rechnungen anzeigen lassen.

\end{document}
