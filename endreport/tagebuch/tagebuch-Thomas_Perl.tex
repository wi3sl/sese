\documentclass[11pt]{scrartcl}

\usepackage{ucs}
\usepackage[utf8x]{inputenc}
\usepackage{ngerman}
\usepackage{amsmath,amssymb,amstext}
\usepackage{graphicx}
\usepackage[automark]{scrpage2}

\pagestyle{scrheadings}

\title{Tagebuch}
\author{Thomas Perl}

\begin{document}

\maketitle

\section{Einträge}

\subsection{4.11.2011 - Meeting}

Das erste Projekttreffen war sehr angenehm. Mein Vorschlag, Python und eine
Web-GUI zu verwenden, wurde nicht angenommen. Stattdessen haben wir uns für
Java und Swing entschieden - damit kann ich leben. Ansonsten bin ich froh,
dass wir Git zur Versionskontrolle und LaTeX für die Dokumentation verwenden.

Die Aufgabenverteilung passt mir, freue mich schon, mit dem Team zu arbeiten.


\subsection{11.11.2011 - Meeting}

Nachdem wir in der ersten Woche mal grundlegende Sachen erledigt haben (und
leider auch in anderen LVAs genug zu tun hatten), haben wir heute schon das
zweite Treffen. Userstories sind gut angelegt, und ich werde in dieser Woche
die Grundstruktur des Projektes anlegen.


\subsection{14.11.2011 - Grundgerüst und GUI}

Heute wurde das Grundgerüst für das Projekt erstellt - ein Java-Projekt mit
dem Namen ,,BlueHotel`` und einer groben Oberfläche. Der Code wurde mit
Vorausschau auf zukünftige Erweiterungen durch Projektmitglieder sehr offen
gehalten, d.h. es wird mit möglichst abstrakten Konstrukten gearbeitet, die
dann einfach durch Subclassing bzw. Generics konkretisiert werden können.

Implementiert wurde unter anderem: Editor (ein Interface, das beschreibt, was
ein "Editor-Objekt" mit Objekten anstellen kann), EditorManager (eine Factory,
die zu einem Objekt von einem bestimmten Typ den richtigen Editor liefert) und
ObjectList (eine grafische Liste von Objekten, die das Anlegen, Bearbeiten und
Löschen von Objekten unterstützt, und sich dabei der vorher genannten Klassen
bedient.

Unzufrieden bin ich mit der technischen Unzulänglichkeit von Java und Swing -
das, was in anderen Toolkits sehr leicht geht, ist in Swing sehr schwierig und
mühsam. Stefan's Tipp, ,,Window Builder Pro`` zu verwenden, hilft mir hier
allerdings, denn mit diesem Eclipse-Plugin kann man das Grundgerüst der GUI
einfach zusammenklicken. Das macht den Code zwar nicht schöner, bringt aber
viel schneller Ergebnisse, auf die man aufbauen kann. Gut, dass wir jemanden
im Team haben, der sich mit diesen Tools auskennt!


\subsection{18.11.2011 - Meeting}

Ein weiteres wöchentliches Meeting hat heute stattgefunden - ich konnte bereits
meine ersten Code-Ergebnisse präsentieren. Bei den Tests hängen wir momentan
noch ein bisschen nach, aber ich bin der Meinung, dass es gut ist, als Basis
für die Diskussionen einmal ein herzeigbares Projekt zu haben.

Habe dem Projektteam erklärt, wie meiner Meinung nach das Editor-Interface zu
funktionieren hat. Insgesamt stehe ich dem Projekt zuversichtlich gegenüber.


\subsection{25.11.2011 - Meeting}

Diese Woche ist von meiner Seite nicht viel weitergegangen, dafür habe ich aber
schon für die nächste Woche einiges eingeplant. Das Erstellen von Rechnungen
wird in den nächsten Tagen zu erledigen sein.


\subsection{30.11.2011 - Eingabevalidierung}

Ein Requirement, dass wir beim letzten Treffen besprochen haben, ist die
Eingabevalidierung. Um den Aufwand wieder so gering wie nötig zu halten, wurde
auch hier wieder sehr abstrakt gearbeitet - so wurde das Editor-Interface um
Funktionen erweitert, die ein Objekt auf Vollständigkeit und Korrektheit
überprüfen können. Im Fehlerfall gibt es auch eine Funktion, die aus einem
Objekt die Fehler als von Menschen lesbaren Text ausgeben kann. Weil das
Handling von Fehlern immer gleich ist (Validierung und wenn fehlgeschlagen,
dann Fehlermeldung anzeigen, ansonsten fortfahren), wurde auch diese Logik in
die Klasse ,,ValidationHandler`` gekapselt.

Für den Kunden-Editor habe ich diese Funktionalität heute einmal komplett
ausprogrammiert - das soll auch als Code-Beispiel für meine Gruppenmitglieder
dienen, wenn diese die Funktionalität für ihre Module implementieren. Eine
Beschreibung des Mechanismus habe ich per E-Mail ans Team geschickt.

Aus der Validierung heraus ergeben sich einige fehlende Funktionalitäten, die
ich als Bug-Reports erstellt habe. Momentan verwalten wir die Bugs als Excel-
Dokument, was meiner Meinung nach suboptimal ist - die Github Issues eignen
sich viel besser dafür. Ich habe das jetzt einmal per E-Mail deutlich zur
Sprache gebracht, und werde auch beim nächsten Treffen versuchen, diese
Änderung durchzubringen, denn auf die Dauer ist das Arbeiten mit dem Excel-
Dokument sehr mühsam, und man hat schlecht überblick über den zeitlichen
Verlauf. In einem Bugtracker (wie Github Issues) sieht man schön, was noch
offen ist, und wie sich der Status von Bugs geändert hat.


\subsection{30.11.2011 - Rechnungs-Assistent}

In der zweiten Programmier-Session des heutigen Tages habe ich den Rechnungs-
Assistenten implementiert. Dieser hilft dabei, eine Liste von Reservierungen
anzuzeigen, und diese dann zu verbuchen. Das wird momentan einfach in der GUI
angezeigt, und ist noch nicht ausimplementiert.

Auch hier stoße ich entweder auf fehlendes Swing-Wissen meinerseits oder auf
Limitierungen von Swing - so musste ich für die Liste der Reservierungen (eine
Multi-Selektions-Liste) eine eigene Klasse erstellen. Es funktioniert, aber
macht den Code etwas unübersichtlicher.


\subsection{1.12.2011 - Einbauen von Invoices}

Heute hatte ich nur wenig Zeit, am Projekt weiter zu arbeiten. Ich habe ein
Icon für die Rechnungen zum Projekt erstellt, um diese in der UI sichtbar zu
machen.


\subsection{2.12.2011 - Meeting}

Ich habe meine Bedenken über die Führung der Bug-Liste als Excel-Dokument dem
Projektteam bekanntgegeben. Nach einiger Diskussion haben wir uns entschieden,
die Bugs wie vorgeschlagen nach Github zu übertragen. Diese Aufgabe übernehme
ich gerne, da es meine Arbeit im Projekt in Zukunft erleichtern wird.


\subsection{2.12.2011 - Bugs zu Issues konvertiert}

Nach dem letzten Meeting wurde beschlosen, dass die Excel-Liste für die Bugs
endlich wegkommt, und wir stattdessen Github Issues verwenden. Darüber bin ich
sehr froh, und es wird meine Motivation für das Projekt steigern. Weiters haben
wir uns auch entschlossen, das Product-Backlog nicht mehr als LaTeX-Datei zu
führen, sondern ins Github-Wiki zu übernehmen.

Beide Änderungen habe ich heute gemacht, und per E-Mail das Team informiert.


\subsection{9.12.2011 - Meeting}

Dieses Meeting war recht kurz, wir haven vorallem in Hinblick auf die
bevorstehende Präsentation schon einige Punkte besprochen. Allgemein ist zu
sagen, dass wir ca. ein Drittel des Projekts fertig haben, vielleicht sogar
ein bisschen mehr.

Insgesamt habe ich das Gefühl, dass wir gut in der Zeit liegen, und gut
miteinander auskommen. Vorallem die wöchentlichen Treffen am Freitag sind ein
guter Abschluss der Uni-Woche, und geben uns Ansporn, in der nächsten Woche
am Projekt weiter zu arbeiten.


\subsection{9.12.2011 - Fehlerüberprüfung beim Löschen}

Beim Löschen von Kunden und Zimmern ist darauf zu achten, dass diese nicht
gelöscht werden dürfen, wenn sie in Reservierungen vorkommen. Dies wurde beim
letzten Meeting besprochen, und ich habe das jetzt implementiert - wobei nun
nur ein Fehler angezeigt wird. Idealerweise sollte hier noch eine bessere
Meldung erscheinen - werde das beim nächsten Meeting zur Sprache bringen.

Seit den letzten 2 Wochen geht es wieder gut voran beim Projekt, die Meetings
jeden Freitag helfen, das Projekt am Laufen zu halten - selbst wenn wir oft nur
eine halbe Stunde oder Stunde über das Projekt sprechen. Der wöchentliche
Austausch ist für so ein Studentenprojekt meiner Meinung nach sehr wichtig.

Ein Online-Treffen (Textchat oder VoIP) würde wohl nicht so viel Motivation
bringen.

Weiters habe ich heute Rechnungs-Infos zum Datenmodell hinzugefügt, und einige
Clean-Ups durchgeführt.


\subsection{13.12.2011 - Lösch-Checks in der Objektliste}

Heute habe ich die Issue 26 behoben - in der Objekt-Liste wird nun immer eine
Bestätigung des Löschens angezeigt (bzw wenn nicht möglich, dann eine andere
Meldung).


\subsection{14.12.2011 - Präsentation}

Für die Präsentation habe ich heute die Präsentation (Demo) mit aktuellem
Stand abgelegt. Bin recht zufrieden mit dem aktuellen Zwischenstand, die
Stimmung im Projekt ist gut - es ist schön, einen kleinen Meilenstein erreicht
zu haben.

\subsection{16.12.2011 - Meeting}

Heute haben wir vorallem zum Thema Rechnungslegung eine offene Punkte
besprochen. Ich hoffe, diese kann ich bald umsetzen. Einige Bugs, die dafür
offen sind, werden von den Projektmitgliedern hoffentlich bald erledigt.

Jetzt haben wir noch Zeit, um einige Nice-to-have Features umzusetzen. Die GUI
ist in der Zwischenzeit schon recht angewachsen, ich hoffe, dass wir am Ende
des Projekts noch etwas Zeit finden, um die GUI etwas aufzuräumen.


\subsection{14.01.2012 - Rechnungs-Generierung}

Heute habe ich den Code für die Erstellung der Rechnungen implementiert. Die
Implementierung wird von den Projektmitgliedern noch getestet werden. Auch hier
war einiges schwerer als erwartet, aber im Endeffekt funktioniert es jetzt
rudimentär sehr gut.

Was hier noch zu erledigen ist, ist etwas Polishing für die Rechnungen und ein
paar Verbesserungen bei der Berechnung. Das sind aber alles Sachen, die dann
später erledigt werden können. Als ,,Feature`` kann das Rechnung Erstellen nun
abgehakt werden.

Das Projekt geht dem Ende zu - es sind nur mehr ein paar Kleinigkeiten zu
erledigen. Wenn es sich ausgeht, will ich mir vor der Endpräsentation noch ein
klein wenig Zeit nehmen, um die GUI und die Rechnungen noch schöner zu machen.


\subsection{17.01.2012 - Polishing}

Heute habe ich fast den ganzen Tag mit Polishing der UI verbracht. Ich bin ein
wenig ausgelaugt, deshalb kein detaillierter Report heute, sondern nur ein
kleiner Überblick über die Änderungen: Suchen/filtern in der Objektliste, mehr
Spacing/Padding in der GUI, Raumbelegungs-Platz als eigenen Menüpunkt in der
Menüleiste, korrigieren der Preis-Eingabe im Zimmer-Editor, Rechnungslegung
verbessert und gepolished.

Ich freue mich schon auf die Präsentation, und finde, dass das Programm mit den
heutigen Änderungen schon sehr professionell aussieht. Habe die Änderungen den
Projektmitgliedern mitgeteilt.


\subsection{18.01.2012 - Endpräsentation}

Heute war die Endpräsentation. Durch das Polishing gestern ist das Endprodukt
heute sehr schön, und Alexander hat sich darum gekümmern, Test-Daten für die
Demo zu erstellen. Freue mich schon auf den Abschluss des Projekts!


\subsection{20.01.2012 - Schreiben des Endreports}

Da ich in der kommenden Woche Uni-Sachen zu erledigen habe, und am Wochenende
danach nicht in Wien bin, habe ich heute meinen Teil des Endreports erledigt.

Rückblickend bin ich sehr zufrieden mit dem Ergebnis, und finde, dass wir ein
herzeigbares und schönes Projekt haben, aber gleichzeitig auch einiges in
Sachen Java und Projektmanagement gelernt haben. Das Arbeiten im Team war gut,
und vorallem die wöchentlichen Meetings steigerten die Motivation.


\end{document}
